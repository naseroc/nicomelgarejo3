\section{¿Qué es un vector?}

Un vector es un segmento de recta dirigido que se caracteriza por tener magnitud o módulo, dirección y sentido. En el plano cartesiano, los vectores se representan gráficamente con una flecha. La \textbf{magnitud} de un vector $\overrightarrow{AB}$ es su longitud, es decir, la distancia entre el inicio $A$ (cola) y el final $B$ (punta de flecha) y se denota de la forma $|\overrightarrow{AB}\,|$.\\

\noindent El \textbf{sentido} del vector está dado por la punta de la flecha que lo representa e indica hacia donde se dirige. La \textbf{dirección} del vector es la orientación o el ángulo que forma la recta que contiene al vector con el eje horizontal del plano cartesiano.

\enunciado{Para representar un vector en el plano cartesiano se deben conocer sus componentes, es decir, las coordenadas $x$ e $y$ que definen al vector $(x, y)$. Para el vector $\overrightarrow{AB}$, sus puntos extremos son $A$ y $B$. Si el punto $A$ tiene coordenadas $(x_1, y_1)$ y el punto $B$ tiene coordenadas $(x_2,y_2)$, las componentes del vector $\overrightarrow{AB}$ están dadas por la diferencia de las abscisas de los puntos final e inicial y la diferencia de las ordenadas de los mismos:
	
	$$\overrightarrow{AB}=(x_2,y_2)-(x_1,y_1)=(x_2 - x_1, y_2-y_1)$$}

\noindent Veamos un ejemplo para comprender la definición anterior: Observa el vector $\overrightarrow{AB}$ en el plano cartesiano:

\begin{center}
	\includegraphics[scale=.5]{img/VQqyFmRA}
\end{center}

\noindent Las coordenadas del punto $A$ son $(-2,-3)$ y las coordenadas del punto $B$ son $(1,4)$, entonces las componentes o coordenadas del vector $\overrightarrow{AB}$ están dadas por:

$$\overrightarrow{AB}=(1,4)-(-2,-3)=(1--2,4--3)=(1+2,4+3)=(3,7)$$
\vspace*{.05cm}

\noindent Se concluye que $\overrightarrow{AB}=(3,7)$. 

\newpage
\noindent Gráficamente puedes verificar que el vector $\overrightarrow{AB}$ representado en el plano cartesiano posee $3$ unidades en la abscisa o eje $x$ y $7$ unidades en la ordenada o eje $y$:

\begin{center}
	\includegraphics[scale=.5]{img/nfSM6S38}
\end{center}

\noindent Observa el vector $(4,3)$ representado en el plano cartesiano:

\begin{center}
	\includegraphics[scale=.45]{img/dyyFl5cw}
\end{center}

\noindent El vector $(4,3)$ tiene sentido noroeste dado por la punta de la flecha, su dirección es 37º con respecto al eje $x$ dado por el ángulo que forma con este y su magnitud es $5$ unidades, es decir, la longitud de la flecha es $5$. ¿Cómo determinamos la magnitud del vector $(4,3)$?\\

\noindent La magnitud o módulo, dado que corresponde a una distancia, siempre será un número positivo y puede ser calculada con el teorema de Pitágoras. En este caso, los catetos del triángulo rectángulo que se forma con las proyecciones de los ejes del plano cartesiano y el vector son $4$ y $3$ unidades como puedes observar en la imagen, mientras que la magnitud o módulo del vector corresponde a la hipotenusa $h$ del triángulo:

$$h^2=4^2+3^2$$
$$h^2=16+9$$
$$h^2=25$$
$$h=\sqrt{25}$$
$$h=5$$

\vspace*{.1cm}

\noindent Por lo tanto, la magnitud o módulo del vector $(4,3)$ es igual a $5$. 

\enunciado{En términos generales, si un vector $\vec v\,$ tiene componentes $(v_x, v_y)$, su magnitud $|\vec v|$ está dada por:
	
	$$|\vec v\,|=\sqrt{v_x^2+v_y^2\;}$$}

\noindent Cuando dos vectores poseen igual magnitud, sentido y dirección, se dice que estos vectores son equivalentes.

\subsection{Multiplicación de un vector por un escalar}

Se denomina escalar a cualquier cantidad perteneciente al conjunto de los números reales que carece de dirección y sentido.

\enunciado{Para multiplicar un vector $(x, y)$ por un escalar $k$ se debe multiplicar la componente $x$ y la componente $y$ del vector por el escalar $k$:
	
	$$k\cdot (x,y)=(k\cdot x,k\cdot y)$$}

\noindent \textbf{Ejemplo}: Sea $\vec{v}= (2,7)$ un vector y $10$ un escalar. ¿Cuál es el resultado de $10\cdot \vec v$?\\

$$10\cdot \vec{v} = 10\cdot (2,7)=(10\cdot 2, 10\cdot 7)=(20,70)$$

\noindent Es decir, $10\cdot \vec v=(20,70)$.\\

\noindent La multiplicación de un vector por un escalar también es denominada ponderación de un vector por un escalar. \\

\noindent Gráficamente, si $k>0$ el resultado corresponde a un vector con la misma dirección y sentido que el vector $(x,y)$, pero con una longitud igual a $k$ veces su longitud. Por otro lado, si $k<0$ el resultado corresponde a un vector con la misma dirección y sentido opuesto al vector $(x,y)$, pero con una longitud igual a $k$ veces su longitud.


\subsection{Adición y sustracción de vectores}

\enunciado{Sean $\vec{v}=(x_1,y_1)$  y  $\vec{w}=(x_2,y_2)$ vectores en el plano cartesiano. Definimos las operaciones adición ($+$) y sustracción ($-$) como:
	
	$$\vec{v}+\vec{w}=(x_1,y_1)+(x_2,y_2)=(x_1+x_2,y_1+y_2)$$
	
	$$\vec{v}-\vec{w}=(x_1,y_1)-(x_2,y_2)=(x_1-x_2,y_1-y_2)$$}

\noindent Nota que en ambas operaciones el resultado también corresponde a un vector del plano cartesiano.\\

\noindent Gráficamente, el vector resultante de la adición $\vec{v}+\vec{w}$ corresponde a la diagonal del paralelogramo formado a partir de los vectores $\vec{v}$  y  $\vec{w}$. Veamos un ejemplo para comprender esta afirmación: \\

\noindent Observa los vectores $\vec{v}=(2,2)$  y  $\vec{w}=(-3,2)$ representados en el plano cartesiano:

\begin{center}
	\includegraphics[scale=.6]{img/Ur1UeD8}
\end{center}

\noindent Puedes verificar que el resultado de la adición de $\vec{v}$  y $\vec{w}$ es el vector $\overrightarrow{v+w}$ representado en el plano cartesiano anterior:

$$\vec{v}+\vec{w}=(2,2)+(-3,2)=(2+-3,2+2)=(-1,4)$$
\vspace*{.1cm}

\noindent Como puedes observar, el vector $\overrightarrow{v+w}$ corresponde a la diagonal del paralelogramo formado a partir de los vectores $\vec{v}$ y $\vec{w}$.\\

\noindent Nota que restar el vector $(x,y)$ es equivalente a sumar el vector $(-x,-y)$, por lo que la interpretación geométrica es idéntica a la dada para la adición de vectores.\\

\noindent \textbf{Ejemplo 1}: Sean $\vec{z}=(3,2)$  y  $\vec{w}=(-5,8)$ vectores, ¿cuál es el resultado de $\vec{z}+\vec{w}$?

$$\vec{z}+\vec{w}=(3,2)+(-5,8)$$
$$\vec{z}+\vec{w}=(3+-5,2+8)$$
$$\vec{z}+\vec{w}=(-2,10)$$

\vspace*{.1cm}

\noindent \textbf{Ejemplo 2}: Sean $\vec p=(-6,-1)$  y  $\vec q=(-2,4)$ vectores, ¿cuál es el resultado de $\vec{p}+\vec{q}$ ?

$$\vec{p}+\vec{q}=(-6,-1)+(-2,4)$$
$$\vec{p}+\vec{q}=(-6+-2,-1+4)$$
$$\vec{p}+\vec{q}=(-8,3)$$

\vspace*{.1cm}

\noindent \textbf{Ejemplo 3}: Sean $\vec{z}=(3,2)$  y  $\vec{w}=(-5,8)$ vectores, ¿cuál es el resultado de $\vec{z}-\vec{w}$?

$$\vec{z}-\vec{w}=(3,2)-(-5,8)$$
$$\vec{z}-\vec{w}=(3--5,2-8)$$
$$\vec{z}-\vec{w}=(3+5,2-8)$$
$$\vec{z}-\vec{w}=(8,-6)$$

\vspace*{.1cm}

\noindent \textbf{Ejemplo 4}: Sean $\vec p=(-6,-1)$  y  $\vec q=(-2,4)$ vectores, ¿cuál es el resultado de $\vec{p}-\vec{q}$ ?

$$\vec{p}-\vec{q}=(-6,-1)-(-2,4)$$
$$\vec{p}-\vec{q}=(-6--2,-1-4)$$
$$\vec{p}-\vec{q}=(-6+2,-1-4)$$
$$\vec{p}-\vec{q}=(-4,-5)$$


\subsection{Traslaciones de figuras geométricas usando vectores}

La traslación es una transformación isométrica que corresponde al movimiento de una figura en una dirección, en un sentido y en una magnitud dada. La dirección, sentido y magnitud del desplazamiento de la figura pueden ser representados por un vector de traslación.\\

\noindent Sea $P$ un punto del plano cartesiano que posee coordenadas $(x,y)$. Si al punto $P$ se le debe aplicar una traslación según el vector $\vec{t}=(t_x, t_y)$, la imagen o punto que se obtiene luego de trasladar el punto $P$ según el vector $\vec t$ posee las coordenadas $(x+t_x, y+t_y)$. Es decir, para trasladar un punto del plano cartesiano usando cierto vector traslación, el punto debe tratarse como vector y se debe resolver la operación $(x,y)+(t_x,t_y)$.\\

\noindent Nota que en este caso el vector obtenido corresponde en realidad a un punto del plano cartesiano.\\

\noindent \textbf{Ejemplo}: Al punto $(6,-4)$ se le aplicó una traslación y se obtuvo el punto $(12,-8)$. Al aplicar al punto $(-3,5)$ la misma traslación, ¿qué punto se obtiene?\\

\noindent Al punto $(6,-4)$ se le aplicó una traslación usando un vector traslación desconocido $(t_x,t_y)$ y se obtuvo el punto $(12,-8)$. Por lo tanto, se debe cumplir que:

$$(6,-4)+(t_x,t_y)=(12,-8)$$
\vspace*{.1cm}

\noindent Es decir:

$$6+t_x=12 \; \Rightarrow\; t_x=12-6 \;\Rightarrow\; t_x=6$$

$$-4+t_y=-8 \; \Rightarrow\; t_y=-8+4=-4$$
\vspace*{.1cm}

\noindent Por lo tanto, concluimos que el vector traslación desconocido originalmente es $(6,-4)$. Para conocer cuál es el punto obtenido al aplicar al punto $(-3,5)$ la traslación según el vector $(6,-4)$ se debe resolver $(-3,5)+(6,-4)$: 

$$(-3,5)+(6,-4)=(-3+6,5+-4)=(3,1)$$
\vspace*{.1cm}

\noindent Concluimos que el punto obtenido al aplicar al punto $(-3,5)$ la traslación según el vector $(6,-4)$ posee coordenadas $(3,1)$.
