\section{Transformaciones isométricas}
Una transformación isométrica es un movimiento en el plano que mantiene la forma y tamaño de una figura, es decir, la figura resultante es \textbf{congruente} a la inicial y se denomina \textbf{figura homóloga} o imagen. La figura inicial, en tanto, se denomina \textbf{figura origen}.\\

Entre las transformaciones isométricas que estudiaremos se encuentran las traslaciones, las rotaciones y las simetrías. Cuando trabajemos con cualquiera de estas tres transformaciones nos será útil acudir a un \textbf{sistema de coordenadas} para poder describir la posición de los diferentes puntos que forman nuestras figuras a transformar.\\

\subsection{Sistema de coordinadas}
Recordemos que el sistema de coordenadas cartesiano está formado por dos rectas perpendiculares numeradas: una horizontal, denominada eje de las abscisas o eje $X$ y otra vertical, denominada eje de las ordenadas o eje $Y$. Estas dos rectas se intersectan en un punto denominado origen que corresponde al $0$ de ambas rectas numéricas.

\begin{figure}[hbt!]
	\centering
	\includegraphics[scale=.3]{img/cartesiano.png}
\end{figure}

La posiciones de los puntos en el plano cartesiano se designan como pares ordenados, por ejemplo el punto $P$ de la figura se escribe como $P(2,-4)$. En esta notación, la primera componente del par ordenado nos indica que el punto $P$ se ubica dos unidades hacia la derecha en el eje de las abscisas. Mientras que la segunda componente del par ordenado nos indica que el punto $P$ se ubica cuatro unidades hacia abajo en el eje de las ordenadas.

\section{Traslaciones}
Trasladar un punto $P(x,y)$ con respecto a un vector traslación $\vec v$, significa mover dicho punto según las componentes que definen al vector traslación. Esto es, el punto se moverá en la misma dirección, en el mismo sentido y en la misma magnitud que el vector $\vec{v}$. En el plano cartesiano se resume a la utilización de las coordenadas del vector $\vec{v} = (a, b)$.\\

Así, el punto $P$ se trasladará $|a|$ unidades hacia la izquierda o hacia la derecha, según indique el signo de \textbf{a}, y $|b|$ unidades hacia arriba o hacia abajo, según indique el signo de \textbf{b}.\\

Para estandarizar este movimiento, basta con darse cuenta que si se aplica una traslación al punto $P(x,y)$ según el vector $\vec{v} = (a,b)$, se obtiene un nuevo. $P'$, cuyas coordenadas estarán dadas por 

\[P' (x+a, y+b)\]

Al punto $P'$ se le llama \textbf{imagen del punto $P$ según la traslación determinada por el vector $\vec{v}$}.\\


En el caso de querer trasladar una figura completa, lo que hacemos es aplicar el vector traslación a todos los puntos que componen la figura. En el caso de un polígono, lo que nos conviene es aplicar la traslación a los vértices de la figura original para obtener la imagen.

\begin{figure}[hbt!]
	\centering
	\includegraphics[scale=.3]{img/traslacion1.png}
\end{figure}


\enunciado{
	\textbf{Ejemplo 1:} Si queremos trasladar el punto $\textcolor{blue}{A(2, -5)}$ con respecto del vector $\textcolor{red}{\vec{v} = (-4,8)}$, lo podemos interpretar como que debemos mover el punto \textbf{P} cuatro cuatro unidades hacia la izquierda (-) y ocho unidades hacia arriba (+) obteniéndose como resultado
	\[P'(\textcolor{blue}{2}+ \textcolor{red}{-4},~ \textcolor{blue}{-5} + \textcolor{red}{8}) = P'(-2,3)\]
	Gráficamente sería esto:
	\begin{center}
		\includegraphics[scale=.4]{img/traslacion-ejemplo.png}
\end{center}}

\enunciado{
	\textbf{Ejemplo 2:} Si al punto $A(2,~4)$ se le aplica una traslación según el vector $\vec v = (6,-2)$, entonces se obtiene el punto:
	\[ A' = (2,~4) + (6, -2) = (8,~ 2) \]
	La representación gráfica de lo anterior es esta:
	\begin{center}
		\includegraphics[scale=.3]{img/traslacion-punto.png}
\end{center}}

\enunciado{
	\textbf{Ejemplo 3:} Si queremos trasladar el $\bigtriangleup{ABC}$ cuyos vértices son los puntos $A(1,~3)$, $B(4,~2)$ y $C(3,~6)$, de acuerdo al vector  $\vec{v}=(-5,-1)$ significa que debemos mover los tres vértices del triángulo 5 unidades hacia la izquierda y 1 unidad hacia abajo, tal como lo muestra la siguiente figura:
	\begin{center}
		\includegraphics[scale=.3]{img/tras1.png}
	\end{center}
	De acuerdo a la imagen podemos notar que para trasladar por ejemplo los vértices del $\bigtriangleup{ABC}$ basta con sumarles a sus pares ordenados el vector traslación, es decir:
	
	\begin{align*}
		A' &= A + \vec v = (1,~3) + (-5,-1) = (-4,~2)\\
		B' &= B + \vec v = (4,~2) + (-5,-1) = (-1,~1)\\
		C' &= C + \vec v = (3,~6) + (-5,-1) = (-2,~5)\\
	\end{align*}
	
	Concluimos entonces que al realizar la traslación del $\bigtriangleup{ABC}$ respecto al vector \newline $\vec{v}=(-5,-4)$ obtenemos el $\bigtriangleup{A'B'C'}$ con coordenadas $A'(-4,~2)$, $B'(-1,~1)$ y $C'(-2,~5)$.}
\newpage

\subsection{Composición de traslaciones}

Cuando hablemos de composición de traslaciones estamos haciendo referencia a la aplicación de más de una traslación consecutiva sobre un objeto.\\

Por ejemplo si a un punto $O(-4,~2)$ le aplicamos la traslación $\vec{v}=(2,-4)$ y sobre esa imagen otra  traslación $\vec{w}=(6,~4)$ obtenemos un punto de coordenadas $O''(4,~2)$.

\begin{equation*}
	\begin{split}
		(-4,~2) + (2,-4) &=(-2,-2)\\
		(-2,-2) + (6,~4) &=(4,~2)
	\end{split}
\end{equation*}

Gráficamente obtenemos lo siguiente:

\begin{figure}[hbt!]
	\centering
	\includegraphics[scale=.4]{img/traslacion-composicion.png}
\end{figure}

Notemos además que aplicar la traslación \textbf{v} y luego la \textbf{w} al punto \textbf{O}, es equivalente a aplicar una única traslación al punto \textbf{O} con un vector traslación igual a la suma del vector \textbf{v} y \textbf{w}.\\

\subsection{Propiedades de la traslación}
\begin{itemize}
	\item Al aplicar una traslación $\vec{v}=(a,b)$ a un punto cualquiera $P(x,y)$ la imagen que se obtiene corresponde al \linebreak punto $P'(x+a,y+b)$.
	\item En la composición de traslaciones el orden en que se aplican a una figura no influyen en el resultado.
	\item Toda composición de traslaciones se puede reducir a una única traslación cuyo vector de traslación corresponde a la suma de cada vector por separado.
\end{itemize}

\section{Simetrías}


Una  simetría o reflexión corresponde a aquellos movimientos que invierten los puntos y las figuras en el plano. Esta transformación isométrica se puede dividir en simetría axial y simetría central. 

\subsection{Simetría axial}
Es una \textbf{transformación isométrica} que mueve cada punto de la figura, de modo que el punto inicial y su homólogo equidistan de una \textbf{recta} llamada \textbf{eje de simetría}. Es importante destacar que si \textbf{A'} es la imagen respecto de una simetría axial del punto \textbf{A} por una recta \textbf{L}, entonces el segmento $\overline{AA'}$ es perpendicular a la recta \textbf{L}.

Por ejemplo, en la figura, el polígono \textbf{A'B'C'D'E'} es la figura homóloga del polígono \textbf{ABCDE}, aplicando una simetría cuyo eje es la recta \textbf{L}.

\begin{figure}[hbt!]
	\centering
	\includegraphics[scale=.28]{img/simetria-axial.png}
\end{figure}


Las distancias de cada vértice de la figura original \textbf{ABCDE} al eje de simetría, son iguales a las de sus vértices homólogos al eje de simetría.

Podemos identificar tres tipos de rectas
\begin{itemize}
	\item Las verticales de ecuación $x = a$ con $a\pertenece \reales$.
	\item Las horizontales de ecuación $y = a$ con $a\pertenece \reales$.
	\item Las oblicuas de ecuación $y = mx + n$ con $m,n \pertenece \reales$ y $m\neq 0$
\end{itemize}

\subsubsection{Si el eje de simetría es vertical ($x = a$)}
Consideremos un punto $P(m,~n)$ al que le aplicamos una simetría axial respecto de $x = a$ y obtenemos el punto $P'(m', n')$. En tal caso se cumplirá que:

\[m' = 2a - m \quad\text{ y }\quad n' = n\]

Esto se debe a que al ser el eje de simetría \textbf{vertical}, el punto hará un movimiento horizontal y, por lo tanto, su ordenada no sufrirá cambios.\\

Por otro lado, las abscisas satisfacen la siguiente condición:

\[\dfrac{m +m'}{2} = a \]

ya que el punto medio entre $P$ y $P'$ pertenece al eje de simetría.

\subsubsection{Si el eje de simetría es horizontal ($y = a$)}

Consideremos un punto $P(m,~n)$ al que le aplicamos una simetría axial respecto de $y = a$ y obtenemos el punto $P'(m', n')$. En tal caso se cumplirá que:

\[m' =  m \quad\text{ y }\quad n' = 2a - n\]

Esto se debe a que al ser el eje de simetría \textbf{horizontal}, el punto hará un movimiento vertical y, por lo tanto, su abscisa no sufrirá cambios.\\

Por otro lado, las ordenadas satisfacen la siguiente condición:

\[\dfrac{n +n'}{2} = a \]

ya que el punto medio entre $P$ y $P'$ pertenece al eje de simetría.

\subsubsection{Si el eje de simetría es oblicuo del tipo $y = x$}
Consideremos un punto $P(m,~n)$ al que le aplicamos una simetría axial respecto de $y = x$ y obtenemos el punto $P'(m', n')$. En tal caso se cumplirá que: $(m', n') = (n, m)$

Esto debido a que será equivalente a rotar el punto en $180^\circ$. También lo podemos asociar la gráfica de la función inversa de una función. Recordemos que en tal caso las gráficas son simétricas respecto a la recta $y = x$.

\subsubsection{Si el eje de simetría es oblicuo del tipo $y = x + a$}
Consideremos un punto $P(m,~n)$ al que le aplicamos una simetría axial respecto de $y = x + a$ y obtenemos el punto $P'(m',n')$. Los pasos para resolver este problema son:

\begin{enumerate}
	\item Trasladamos la recta y el punto respecto del vector traslación $\vec{v} = (0, - a)$. Obtendremos una nueva recta $y = x$ con la misma pendiente que la original pero que pasa por el origen. Además el punto $P(m,n)$ pasará a $P''(m, +n - a)$.
	\item Hacemos la simetría del punto $P''$ respecto a la recta $y = x$ según lo visto en el caso anterior. Obtendremos el punto $P'''(n-a, m)$.
	\item Trasladamos el punto $P'''(n-a, m)$ respecto del vector traslación $-\vec{v} = (0, a)$ obteniendo el punto buscado: 
	
	\[P'(m', n') = (n - a, m + a)\]
\end{enumerate} 

\enunciado{
	\textbf{Ejemplo 1: } ¿Cuál es el resultado de aplicar una simetría axial al punto $P(2,~4)$ respecto de la recta $x = 5$ ?\\
	
	Si se refleja el punto $P(2,~4)$ con respecto a la recta $x = 5$, se obtiene el punto $(8,~4)$, ya que la ordenada (4) se mantiene fija,  mientras que la abscisa del punto $P'$ debe ser un valor \textbf{x} tal que promediado con 2 resulte 5
	
	\begin{align*}
		\dfrac{x + 2}{2} &= 5\\
		x &= 10 - 2\\
		x &= 8
	\end{align*}
	
	En este caso el valor de la abscisa de $P'$ es 8 y el punto es $P'(8,~4)$.\\
	
	
	
	\textbf{Ejemplo 2: } ¿Cuál es el resultado de aplicar una simetría axial al punto $P(2,~4)$ respecto de la recta $y = 5$ ?\\
	
	Si se refleja el punto $P(2,~4)$ con respecto a la recta $y = 5$, se obtiene el punto $(2,~6)$, ya que la abscisa (2) se mantiene fija, mientras que la ordenada del punto $P'$ debe ser un valor \textbf{y} tal que promediado con 4 resulte 5
	
	\begin{align*}
		\dfrac{y + 4}{2} &= 5\\
		y &= 10 - 4\\
		y &= 6
	\end{align*}
	
	En este caso el valor de la ordenada de $P'$ es 6 y el punto es $P'(2,~6)$.
	
}
\subsection{Composición de simetrías axiales}

Cuando hablemos de composición de simetrías axiales estamos haciendo referencia a la aplicación  de dos simetrías axiales a una figura de forma consecutiva, es decir, una después de la otra.\\

Por ejemplo, si reflejamos el punto $A$ respecto a la recta $L$ obtenemos $A'$. Luego $A'$ le aplicamos una simetría axial respecto de $M$  y obtenemos finalmente $A''$.

\begin{figure}[hbt!]
	\centering
	\includegraphics[scale=.5]{img/simsim.png}
\end{figure}


Por otro lado, si al mismo punto $A$ le aplicamos las simetrías en orden inverso, es decir primero reflejamos respecto a la recta $M$ y luego respecto a la recta $L$ obtenemos el punto circular $A^{\circ\circ}$ que muestra la figura, el cual es distinto al punto $A''$.\\

\enunciado{A partir de lo anterior, podemos darnos cuenta que en la composición de simetrías axiales \textbf{no es conmutativa}, es decir, sí importa el orden de aplicación.}

\subsection{Simetría central}
Una \textbf{simetría central} de un punto $P(x,~y)$ efectuada con respecto a un punto fijo $C(a,~b)$, llamado \textbf{centro de simetría}, es una transformación isométrica que asocia un punto $P(x,~y)$ con un punto $P'(x',~y')$ tal que: 

\[\dfrac{x+x'}{2} = a \quad \text{ e }\quad \dfrac{y + y'}{2} = b  \]

Esto se debe a que el punto $P'(x',~y')$ y $P(x,~y)$ están a la misma distancia del punto $C(a,~b)$ y, además, $P'$ pertenece a la recta que contiene al segmento $\overline{PC}$.\\

Por ejemplo, si tenemos el triángulo $PQR$ y le aplicamos una simetría central respecto de $C$ obtenemos el triángulo $P'Q'R'$
\begin{figure}[hbt!]
	\centering
	\includegraphics[scale=.28]{img/simetria-central.png}
\end{figure}

\textbf{Aplicar una simetría central equivale a rotar en $180^\circ$, considerando como centro de rotación el centro de simetría.}



\subsection{Rotaciones}
Una  rotación es una transformación isométrica que mueve los puntos según un punto fijo llamado \textbf{centro de rotación} (D) y un ángulo de rotación ($\alpha$). Generalmente se dennota $R_{(D,~\alpha)}$. Las rotaciones en sentido \textbf{positivo} se realizan en \textbf{contra} del movimiento de los punteros del reloj (antihorario), mientras que las que tienen sentido \textbf{negativo}, se realizan \textbf{a favor} del movimiento de los punteros del reloj.\\

En la figura, $A'B'C'$ es la imagen de $ABC$ luego de aplicarle una rotacicón $R_{(D,~\alpha ^\circ)}$

\begin{figure}[hbt!]
	\centering
	\includegraphics[scale=.35]{img/rotacion-ejemplo.png}
\end{figure}
\newpage
\subsubsection{Rotaciones notables}

Para cualquier punto $(x,y)$ que se rota en torno al origen del eje de coordenadas en los ángulos de $90^\circ$, $180^\circ$, $270^\circ$ y $360^\circ$ se tiene que:

\begin{center}
	\begin{tabular}{|c|c|c|c|c|}\hline
		\textbf{Punto inicial} & \textbf{$90\text{°}$} & \textbf{$180\text{°}$} & \textbf{$270\text{°}$} & \textbf{$360\text{°}$} \\ \hline
		$(x,~y)$ & $(-y,~x)$ & $(-x,-y)$ & $(y,-x)$ & $(x,~y)$ \\ \hline
	\end{tabular}
\end{center}

La regla es que por cada rotación en 90 grados, invertimos el orden de las coordenadas, y a la primera coordenada resultante le invertimos el signo.

\subsubsection{Rotaciones con centro que no es el origen}
Para rotar un punto cualquiera, con respecto a cualquier punto del plano (que no sea el origen), se puede seguir el procedimiento que se describe a continuación.\\

Supongamos que queremos rotar el punto $P(x,y)$, con respecto al punto $C(h,k)$, en un ángulo de $90^\circ$ y que resultado esa rotación será $Q(a,b)$. Podemos seguir estos pasos:

\begin{enumerate}
	\item Se trasladan todos los puntos de tal manera que nuestro centro de rotación sea $(0,0)$ que es el punto al cual ya sabemos rotar. Para esto trasladamos según el vector $\vec{v} = (-h, -k)$, así obtenemos los puntos $P'(x-h, y-k)$ y $C'(0,0)$.
	
	\item Se rota ahora $P'$ respecto de $C'$ según el ángulo pedido de $90^\circ$ que según las reglas anteriores será $P''(k-y, x-h)$.
	
	\item Una vez hecha la rotación anterior, lo que nos falta es volver al sistema de coordenadas inicial (es equivalente a deshacer la traslación inicial). Para esto utilizamos el vector $\vec{v} = (h,k)$. Así $P'$ vuelve a ser $P(x,y)$ y $C'$ vuelve a ser $C$.
	
	\[Q(a,b) = (k-y+h, x-h+k)\]
	
\end{enumerate}