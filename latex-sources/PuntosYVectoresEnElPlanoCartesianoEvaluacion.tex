\section{Ejercicios}

\begin{enumerate}[label=\large{\textbf{\arabic*.}}, itemsep = 0.15cm, topsep = 0.5cm]
	
	\parbox{1\linewidth}{\item En el plano de la figura, se muestra el polígono ABCD, ¿cuál(es) de las siguientes afirmaciones es (son) verdadera(s)? \autorpregunta{DEMRE 2005}
		\begin{enumerate}[label={\Roman*)}, itemsep = 0.5cm, topsep = 0.5cm, leftmargin = 2.4cm]
			\item El perímetro del polígono es $8\sqrt{2}$.
			\item Cada diagonal del polígono mide 4.
			\item El área del polígono es $4\sqrt{2}$.
		\end{enumerate}
		\begin{minipage}[t][3cm][t]{0.5\linewidth}
			\vspace{0.25cm}
			\begin{enumerate}[label={\Alph*)}, itemsep = 0.15cm, topsep = 0.5cm]
				\item Solo I
				\item Solo II
				\item Solo I y II
				\item Solo II y III
				\item I, II y III
			\end{enumerate}
		\end{minipage}
		\begin{minipage}[t][1cm][t]{0.49\linewidth}
			\vspace{0cm} % Se puede jugar con el vspace y el hspace para acomodar
			\hspace{-2cm} % la figura, dependiendo de su tamaño
			\includegraphics[scale=0.5]{img/p01gpunto} % El nombre del archivo, también se puede cambiar la escala
	\end{minipage}}
	
	\vspace*{1.2cm}
	
	
	\parbox{1\linewidth}{ \item Al ubicar los puntos A($-1$, $-2$), B(5, $-2$) y C(5,3), en el sistema de ejes coordenados, se puede afirmar que:
		\begin{enumerate}[label={\Roman*)}, itemsep = 0.5cm, topsep = 0.5cm, leftmargin = 2.4cm]
			\item $\overline{\mbox{AB}}\perp\overline{\mbox{BC}}$
			\item $\overline{\mbox{AB}}$ es paralelo al eje $x$.
			\item (0,5) es un punto del trazo BC.
		\end{enumerate}
		Es (son) correcta(s) \autorpregunta{DEMRE 2007}
		\begin{enumerate}[label={\Alph*)}, itemsep = 0.15cm, topsep = 0.5cm]
			\item Solo II
			\item Solo I y II
			\item Solo I y III
			\item Solo II y III
			\item I, II y III
	\end{enumerate}}
	
	\newpage
	\vspace*{-1.5cm}
	
	\parbox{1\linewidth}{ \item En la figura están representados los vectores $\vec{a}$ y $\vec{b}$. ¿Cuál(es) de los gráficos presentados en I), en II) y en III) representa(n) la suma de estos dos vectores? 
		
		\autorpregunta{DEMRE 2011}
		\vspace*{-.3cm}
		\begin{center}
			\includegraphics[scale=.3]{img/p03gpunto}
		\end{center}
		\begin{enumerate}[label={\Alph*)}, itemsep = 0.15cm, topsep = 0.3cm]
			\item Solo I
			\item Solo II
			\item Solo I y II
			\item Solo I y III
			\item Solo II y III
	\end{enumerate}}
	
	
	\parbox{1\linewidth}{ \item Si P es el conjunto de todos los puntos del plano de la forma (3, $y$) y S es el conjunto de todos los puntos del plano de la forma ($x$, 2), entonces el único punto común entre los conjuntos P y S es\autorpregunta{DEMRE 2012}
		\begin{enumerate}[label={\Alph*)}, itemsep = 0.15cm, topsep = 0.5cm]
			\item (5, 1)
			\item (3, 2)
			\item (2, 3)
			\item  (1,$-1$)
			\item  (0, 0)
	\end{enumerate}}
	
	
	\parbox{1\linewidth}{\item En el plano cartesiano de la figura, se ubican los vectores $\vec{a}$ y $\vec{b}$ . ¿Cuál(es) de las siguientes afirmaciones es (son) verdadera(s)?	\autorpregunta{DEMRE 2012}
		\begin{enumerate}[label={\Roman*)}, itemsep = 0.5cm, topsep = 0.5cm, leftmargin = 2.4cm]
			\item $3\vec{a}=(12,15)$
			\item $\vec{a}+\vec{b}=(7,1)$
			\item $-\vec{b}=(-3,-4)$
		\end{enumerate}
		\begin{minipage}[t][3cm][t]{0.5\linewidth}
			\begin{enumerate}[label={\Alph*)}, itemsep = 0.15cm, topsep = 0.3cm]
				\item Solo I
				\item Solo I y II
				\item Solo I y III
				\item Solo II y III
				\item I, II y III
			\end{enumerate}
		\end{minipage}
		\begin{minipage}[t][1cm][t]{0.49\linewidth}
			\vspace{-2cm} % Se puede jugar con el vspace y el hspace para acomodar
			\hspace{-1cm} % la figura, dependiendo de su tamaño
			\includegraphics[scale=0.25]{img/p05gpunto} % El nombre del archivo, también se puede cambiar la escala
	\end{minipage}}
	
	
	\parbox{1\linewidth}{ \item Si $\vec{a}=\left(\dfrac{3}{2}, 6\right)$ y $\vec{b}=\left(-\dfrac{3}{2},-6\right)$, entonces $4\vec{a}-2\vec{b}$ es igual a \autorpregunta{DEMRE 2014}
		\begin{enumerate}[label={\Alph*)}, itemsep = 0.15cm, topsep = 0.5cm]
			\item (3, 0)
			\item (9, 0)
			\item (9, 12)
			\item (3, 12)
			\item (9, 36)
	\end{enumerate}}
	
	
	\parbox{1\linewidth}{ \item Dados $\vec{v} = (m, 2)$ y $\vec{u} = (3, 4)$, ¿cuál de los siguientes números puede ser el valor de $m$ para que la longitud de $\vec{v}$ sea el doble de la longitud de $\vec{u}$? \autorpregunta{DEMRE 2015}
		\begin{enumerate}[label={\Alph*)}, itemsep = 0.15cm, topsep = 0.5cm]
			\item $\sqrt{96}$
			\item $\sqrt{104}$
			\item $\sqrt{46}$
			\item $\sqrt{21}$
			\item 1
	\end{enumerate}}
	
	\parbox{1\linewidth}{ \item Dos vértices de un cuadrado son los puntos (0, 0) y (3, 4). ¿Cuál de los siguientes puntos \textbf{NO} puede ser otro de los vértices del cuadrado? \autorpregunta{DEMRE 2015}
		\begin{enumerate}[label={\Alph*)}, itemsep = 0.15cm, topsep = 0.5cm]
			\item (4, $-3$)
			\item (7, 1)
			\item (5, 0)
			\item ($-4$, 3)
			\item ($-1$, 7)
	\end{enumerate}}
	
	\parbox{1\linewidth}{ \item Sean los puntos M y P de coordenadas (2, 3) y (5, $p$), respectivamente, con P en el cuarto cuadrante. Si la distancia entre estos puntos es 7 unidades, entonces el valor de $p$ es \autorpregunta{DEMRE 2015}
		\begin{enumerate}[label={\Alph*)}, itemsep = 0.15cm, topsep = 0.5cm]
			\item $3-2\sqrt{10}$
			\item $3+2\sqrt{10}$
			\item $\sqrt{31}$
			\item $-\sqrt{31}$
			\item $-\sqrt{67}$
	\end{enumerate}}
	
	
	
	\parbox{1\linewidth}{\item En un sistema de ejes coordenados se puede determinar el radio de una circunferencia, si se conoce: 
		
		\autorpregunta{DEMRE 2015}
		\begin{enumerate}[label={(\arabic*)}, leftmargin = 3cm, topsep = 0.5cm]
			\item El centro de la circunferencia y un punto de ella.
			\item Dos puntos de la circunferencia.
		\end{enumerate}
		\begin{enumerate}[label={\Alph*)}, itemsep = 0.15cm, topsep = 0.5cm]
			\item (1) por sí sola
			\item (2) por sí sola
			\item Ambas juntas, (1) y (2)
			\item Cada una por sí sola, (1) ó (2)
			\item Se requiere información adicional
	\end{enumerate}}
	
	
	\parbox{1\linewidth}{\item Por los puntos A y B de la figura se trazan paralelas al eje $x$ y al eje $y$ formándose un polígono. ¿Cuál(es) de las siguientes afirmaciones es (son) verdadera(s)?	\autorpregunta{DEMRE 2016}\codigoyoutube{\ding{42}}{c1m1VjFDKZA}
		\begin{enumerate}[label={\Roman*)}, itemsep = 0.5cm, topsep = 0.5cm, leftmargin = 2.4cm]
			\item El polígono es un cuadrado.
			\item $\mbox{AB}=5\sqrt{2}$
			\item El perímetro del polígono es 20.
		\end{enumerate}
		\begin{minipage}[t][3cm][t]{0.5\linewidth}
			\begin{enumerate}[label={\Alph*)}, itemsep = 0.15cm, topsep = 0.3cm]
				\item Solo I
				\item Solo III
				\item Solo I y II
				\item Solo I y III
				\item I, II y III
			\end{enumerate}
		\end{minipage}
		\begin{minipage}[t][1cm][t]{0.49\linewidth}
			\vspace{-1cm} % Se puede jugar con el vspace y el hspace para acomodar
			\hspace{1cm} % la figura, dependiendo de su tamaño
			\includegraphics[scale=0.3]{img/p11gpunto} % El nombre del archivo, también se puede cambiar la escala
	\end{minipage}}
	
	\vspace*{1cm}
	
	\parbox{1\linewidth}{ \item Si $a < 0$, entonces la magnitud del vector $(-a)(a^2, a^2)$ es \autorpregunta{DEMRE 2016}\codigoyoutube{\ding{42}}{SlIHVSLb4ys}
		\begin{enumerate}[label={\Alph*)}, itemsep = 0.15cm, topsep = 0.5cm]
			\item $\sqrt{2}a^2$
			\item $-a^5$
			\item $-a$
			\item $2a^3$
			\item $-\sqrt{2}a^3$
	\end{enumerate}}
	
	\parbox{1\linewidth}{ \item Si P y Q son dos puntos ubicados en el eje de las ordenadas que están a una distancia de $\sqrt{10}$ del punto (1, 2), entonces la distancia entre P y Q es \autorpregunta{DEMRE 2016}\codigoyoutube{\ding{42}}{L4-X0h0S7Bk}
		\begin{enumerate}[label={\Alph*)}, itemsep = 0.15cm, topsep = 0.5cm]
			\item 4
			\item 6
			\item $2\sqrt{6}$
			\item 10
			\item $2\sqrt{10}$
	\end{enumerate}}
	
	\parbox{1\linewidth}{\item Si en el plano cartesiano de la figura adjunta se representan $\vec{v}$ y $\vec{w}$, entonces $(2\vec{v}-\vec{w})$  es
		
		\autorpregunta{DEMRE 2017}\codigovimeo{\ding{42}}{176262795}\\	
		\begin{minipage}[t][3cm][t]{0.5\linewidth}
			\begin{enumerate}[label={\Alph*)}, itemsep = 0.15cm, topsep = 0.3cm]
				\item (5, 9)
				\item  (3, 9)
				\item  ($-4$, 0)
				\item (9, 5)
				\item ninguno de los vectores anteriores.
			\end{enumerate}
		\end{minipage}
		\begin{minipage}[t][1cm][t]{0.49\linewidth}
			\vspace{-0.5cm} % Se puede jugar con el vspace y el hspace para acomodar
			\hspace{1cm} % la figura, dependiendo de su tamaño
			\includegraphics[scale=0.25]{img/p14gpunto} % El nombre del archivo, también se puede cambiar la escala
	\end{minipage}}
	
	\vspace*{2cm}
	
	\parbox{1\linewidth}{ \item Si las coordenadas de los vértices de un triángulo son (4, 0), (12, 0) y (12, 8), ¿cuál es el área del triángulo, en unidades cuadradas? \autorpregunta{DEMRE 2017}\codigovimeo{\ding{42}}{176262785}
		\begin{enumerate}[label={\Alph*)}, itemsep = 0.15cm, topsep = 0.5cm]
			\item 32 
			\item 48 
			\item 96 
			\item 64
			\item $16\sqrt{2}$
	\end{enumerate}}
	
	\parbox{1\linewidth}{\item En la figura adjunta el triángulo ABC tiene sus catetos paralelos a los ejes coordenados. Si $\mbox{AB} = 2\sqrt{10}$ unidades y $p > 0$, entonces las coordenadas del punto medio de $\overline{\mbox{AB}}$ son 
		
		\autorpregunta{DEMRE 2017}\codigovimeo{\ding{42}}{176542638}\\	
		\begin{minipage}[t][3cm][t]{0.5\linewidth}
			\begin{enumerate}[label={\Alph*)}, itemsep = 0.15cm, topsep = 0.3cm]
				\item (3, 1)
				\item (8, 3)
				\item (14, 3)
				\item (3, 3)
				\item (4, 3)
			\end{enumerate}
		\end{minipage}
		\begin{minipage}[t][1cm][t]{0.49\linewidth}
			\vspace{-0.5cm} % Se puede jugar con el vspace y el hspace para acomodar
			\hspace{-2cm} % la figura, dependiendo de su tamaño
			\includegraphics[scale=0.3]{img/p16gpunto} % El nombre del archivo, también se puede cambiar la escala
	\end{minipage}}
	
	
	\parbox{1\linewidth}{\item Se pueden determinar las coordenadas del extremo de un vector dado $\vec{u}$, que tiene la misma dirección y origen que $\vec{v}$ de la figura adjunta, si se sabe que: \autorpregunta{DEMRE 2017}\codigovimeo{\ding{42}}{176542643}	
		\begin{enumerate}[label={(\arabic*)}, leftmargin = 3cm, topsep = 0.5cm]
			\item $\vec{u}$ y $\vec{v}$ tienen el mismo sentido.
			\item El módulo de $\vec{u}$ es igual al doble del módulo de $\vec{v}$.
		\end{enumerate}
		\begin{minipage}[t][3cm][t]{0.5\linewidth}
			\begin{enumerate}[label={\Alph*)}, itemsep = 0.15cm, topsep = 0.5cm]
				\item (1) por sí sola
				\item (2) por sí sola
				\item Ambas juntas, (1) y (2)
				\item Cada una por sí sola, (1) ó (2)
				\item Se requiere información adicional
			\end{enumerate}
		\end{minipage}
		\begin{minipage}[t][1cm][t]{0.49\linewidth}
			\vspace{0cm} % Se puede jugar con el vspace y el hspace para acomodar
			\hspace{0cm} % la figura, dependiendo de su tamaño
			\includegraphics[scale=0.5]{img/p17gpunto} % El nombre del archivo, también se puede cambiar la escala
	\end{minipage}}
	
	\vspace*{1cm}
	
	\parbox{1\linewidth}{ \item Considere los vectores $\vec{p}$(6, $-4$), $\vec{q}$(2, 9), $\vec{r}$(5, $-2$) y $\vec{s}$(3, 7). ¿Cuál(es) de las siguientes afirmaciones es (son) verdadera(s)?\autorpregunta{DEMRE 2018}\codigovimeo{\ding{42}}{242985521}	
		\begin{enumerate}[label={\Roman*)}, itemsep = 0.5cm, topsep = 0.5cm, leftmargin = 2.4cm]
			\item El vector $(\vec{q}-\vec{r})$ se encuentra en el segundo cuadrante.
			\item El vector $(\vec{s}-2\vec{p})$ se encuentra en el tercer cuadrante.
			\item $\vec{p}+\vec{q}=\vec{r}+\vec{s}$
		\end{enumerate}
		\begin{enumerate}[label={\Alph*)}, itemsep = 0.15cm, topsep = 0.3cm]
			\item Solo I
			\item Solo I y II
			\item Solo I y III
			\item Solo II y III
			\item I, II y III
	\end{enumerate}}
	
	
	\parbox{1\linewidth}{ \item Considere los puntos P($x, y$), Q($-x, -y$) y O(0, 0), con $x$ e $y$ números enteros. ¿Cuál(es) de las siguientes afirmaciones es (son) \textbf{siempre} verdadera(s)? \autorpregunta{DEMRE 2018}\codigovimeo{\ding{42}}{242985529}	
		\begin{enumerate}[label={\Roman*)}, itemsep = 0.5cm, topsep = 0.5cm, leftmargin = 2.4cm]
			\item La distancia entre P y Q es 0.
			\item La distancia entre P y O es la misma que la distancia entre Q y O. 
			\item Los puntos P, Q y O son colineales.
		\end{enumerate}
		\begin{enumerate}[label={\Alph*)}, itemsep = 0.15cm, topsep = 0.3cm]
			\item Solo I
			\item Solo II
			\item Solo III
			\item Solo I y III
			\item Solo II y III
	\end{enumerate}}
	
	\newpage
	\vspace*{-1.2cm}
	
	\parbox{1\linewidth}{ \item ¿Cuál es el radio de la circunferencia que tiene como centro el punto ($-1$, 1) y el punto ($-5, -2$) pertenece a ella? \autorpregunta{DEMRE 2018}\codigovimeo{\ding{42}}{244240026}	
		\begin{enumerate}[label={\Alph*)}, itemsep = 0.15cm, topsep = 0.5cm]
			\item $3\sqrt{5}$ unidades 
			\item 5 unidades
			\item 7 unidades
			\item $\sqrt{37}$ unidades 
			\item $\sqrt{17}$ unidades
	\end{enumerate}}
	
	\parbox{1\linewidth}{ \item Sean los vectores $\vec{v}$(7, $-5$) y $\vec{m} = \vec{v} - \vec{u}$, tal que $\vec{m}$ está en el segundo cuadrante. ¿Cuál de los siguientes vectores podría ser $\vec{u}$? \autorpregunta{DEMRE 2019}\codigovimeo{\ding{42}}{287459787}	
		\begin{enumerate}[label={\Alph*)}, itemsep = 0.15cm, topsep = 0.5cm]
			\item ($-6$, 8)
			\item (8, 6)
			\item ($-8$, 6)
			\item (8, $-6$)
			\item ($-8$, $-6$)
	\end{enumerate}}
	
	\parbox{1\linewidth}{ \item Considere el rectángulo ABCD, donde tres de sus vértices son A($b, b$), B($a, b$) y C$\left(a,-\dfrac{1}{2}b\right)$, con $a$ y $b$ números reales tal que $ab<0$ y $a<b$. ¿Cuál de las siguien\-tes expresiones representa \textbf{siempre} el área de este rectángulo? \autorpregunta{DEMRE 2019}\codigovimeo{\ding{42}}{289320152}	
		\begin{enumerate}[label={\Alph*)}, itemsep = 0.15cm, topsep = 0.5cm]
			\item $\dfrac{3b(b-a)}{2}$
			\item $\dfrac{3b(b+a)}{2}$
			\item $\dfrac{b(b+a)}{2}$
			\item $\dfrac{b(a-b)}{2}$
			\item $\dfrac{3b(a-b)}{2}$
	\end{enumerate}}
	
	\parbox{1\linewidth}{ \item Considere los puntos del plano cartesiano A(4, 5), B(8, 2) y C(12, $p$), con $p>0$ . Si la distancia entre A y C es el doble que la distancia entre A y B, ¿cuál es el valor de $p$? 
		
		\autorpregunta{DEMRE 2019}\codigovimeo{\ding{42}}{287460255}	
		\begin{enumerate}[label={\Alph*)}, itemsep = 0.15cm, topsep = 0.5cm]
			\item 1
			\item 7
			\item 11
			\item $\sqrt{51}$
			\item Ninguno de los anteriores
	\end{enumerate}}
	
	
	\parbox{1\linewidth}{ \item ¿Cuál de los siguientes puntos del plano cartesiano está más distante del
		punto (2, 3)? 
		
		\autorpregunta{DEMRE 2020}
		\begin{enumerate}[label={\Alph*)}, itemsep = 0.15cm, topsep = 0.5cm]
			\item ($-1$, 3)
			\item (4, 5)
			\item (0, 4)
			\item (5, 4)
			\item (2, 6)
	\end{enumerate}}
	
	\parbox{1\linewidth}{ \item Considere los vectores $\vec{u}$ y $\vec{v}$ tal que $\vec{u}+\vec{v}=(-4,-1)$ y $2\vec{u}-\vec{v}=(10,-11)$. ¿Cuál de las siguientes coordenadas corresponde a $\vec{v}$? \autorpregunta{DEMRE 2020}
		\begin{enumerate}[label={\Alph*)}, itemsep = 0.15cm, topsep = 0.5cm]
			\item (2, $-4$)
			\item  ($-6$, 3)
			\item (6, $-12$)
			\item $\left(-\dfrac{26}{3}, \dfrac{7}{3}\right)$
			\item $\left(-9, \dfrac{9}{2}\right)$
	\end{enumerate}}
	
	\parbox{1\linewidth}{ \item Un vértice de un rombo de perímetro 20 unidades, está en A(4, 3). Si se sabe que, exactamente, dos vértices del rombo tienen abscisa negativa y uno de estos dos tiene la ordenada igual a la ordenada del punto A, ¿cuál de las siguientes coordenadas corres\-ponde a uno de los vértices mencionados? 
		
		\autorpregunta{DEMRE 2020}
		\begin{enumerate}[label={\Alph*)}, itemsep = 0.15cm, topsep = 0.5cm]
			\item ($-4$, 3)
			\item ($-2$, 3)
			\item (4, $-3$)
			\item (4, $-2$)
			\item ($-1$, 3)
	\end{enumerate}}
	
	
	\parbox{1\linewidth}{ \item Un cuadrilátero tiene como vértices los puntos A(0, 0), B($a$, 0), C($a, b$) y D(0, $b$). Si $a$ y $b$ son números reales positivos, ¿cuál de las siguientes expresiones representa \textbf{siempre} la suma de las medidas de sus diagonales? \autorpregunta{DEMRE 2020}
		\begin{enumerate}[label={\Alph*)}, itemsep = 0.15cm, topsep = 0.5cm]
			\item $\sqrt{2(a^2+b^2)}$
			\item $2\sqrt{a^2+b^2}$
			\item $2(a+b)$
			\item $2(a^2+b^2)$
			\item $\sqrt{2}(a+b)$
	\end{enumerate}}
	
	
	
	\parbox{1\linewidth}{ \item Considere los vectores $\vec{u}=(2,-1)$, $\vec{v}=(-8,5)$ y $\vec{w}(-5,-3)$. ¿Cuál de los siguientes vectores corresponde al vector $(2\vec{u}-\vec{v}+3\vec{w})$?\autorpregunta{DEMRE 2021}
		\begin{enumerate}[label={\Alph*)}, itemsep = 0.15cm, topsep = 0.5cm]
			\item $(-3,-6)$
			\item $(-3,1)$
			\item $(-3, -16)$
			\item $(-19,-6)$
			\item $(-19,-16)$
	\end{enumerate}}
	
	
\end{enumerate}

\section{Claves}

\begin{tabular}{|c|c|c|c|c|c|c|c|c|c|c|c|c|c|c|c|c|c|c|c|} %En este caso hay dos columnas, de 4.25 y 1.75 cm.
	\hline
	\cellcolor{calipso2!90}01 & 	\cellcolor{calipso2!90}02 & 	\cellcolor{calipso2!90}03&	\cellcolor{calipso2!90}04&	\cellcolor{calipso2!90}05&	\cellcolor{calipso2!90}06&	\cellcolor{calipso2!90}07&	\cellcolor{calipso2!90}08&	\cellcolor{calipso2!90}09&	\cellcolor{calipso2!90}10 &	\cellcolor{calipso2!90}11 &	\cellcolor{calipso2!90}12 &	\cellcolor{calipso2!90}13&	\cellcolor{calipso2!90}14&	\cellcolor{calipso2!90}15&	\cellcolor{calipso2!90}16&	\cellcolor{calipso2!90}17&	\cellcolor{calipso2!90}18&	\cellcolor{calipso2!90}19&	\cellcolor{calipso2!90}20\\ \hline %Datos de ejemplo
	C&B&D&B&B&E&A&C&A&A&E&E&B&A&A&B&C&C&E &B\\ \hline
	\cellcolor{calipso2!90}21 & 	\cellcolor{calipso2!90}22 & 	\cellcolor{calipso2!90}23&	\cellcolor{calipso2!90}24&	\cellcolor{calipso2!90}25&	\cellcolor{calipso2!90}26&	\cellcolor{calipso2!90}27&	\cellcolor{calipso2!90}28&	\cellcolor{calipso2!90}29&	\cellcolor{calipso2!90}30 &	\cellcolor{calipso2!90}31 &	\cellcolor{calipso2!90}32 &	\cellcolor{calipso2!90}33&	\cellcolor{calipso2!90}34&	\cellcolor{calipso2!90}35&	\cellcolor{calipso2!90}36&	\cellcolor{calipso2!90}37&	\cellcolor{calipso2!90}38&	\cellcolor{calipso2!90}39&	\cellcolor{calipso2!90}40\\ \hline %Datos de ejemplo
	D&A&C&D&B&E&B&C&&&&&&&&&&& &\\ \hline
	
\end{tabular}